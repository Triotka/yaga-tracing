\documentclass{easychair}

\newcommand{\todo}[1]{$\langle\langle$#1 $\rangle\rangle$}
\usepackage{hyperref}
\usepackage{xspace}

\newcommand{\qflra}{QF\_LRA}


\newcommand{\yaga}{\textsc{Yaga}\xspace}

\title{The \yaga SMT Solver in SMT-COMP 2023}
\author{
Martin Blicha\inst{1,2} \and
Drahom{\' i}r Han{\' a}k\inst{1} \and
Jan Kofro{\v n}\inst{1}\\
}
\institute{Charles University, Prague, Czech Republic \and Universit{\`a} della Svizzera italiana, Lugano,
Switzerland}
\date{}
\titlerunning{The \yaga SMT Solver}
\authorrunning{Blicha et al.}
\begin{document}
\maketitle

\section{Overview}

\yaga is a new SMT solver developed at Charles University with the goal to investigate alterna-
tives to the dominant CDCL(T) framework for SMT solving. The solver implements the Model
Constructing Satisfiability Calculus (MCSAT)~\cite{de_moura_model-constructing_2013,jovanovic_design_2013}.
It currently has implementation of plugins for Boolean and rational variables which can be used to decide problems in quantifier-free linear real arithmetic. The Boolean plugin uses the typical mechanism of watched literals~\cite{Chaff} to perform Boolean constraint propagation. The plugin for linear real arithmetic uses a similar mechanism of watched variables to keep track of variable bounds~\cite{jovanovic_design_2013}. The last checked variable in each clause or a linear constraint is cached. Search for a non-falsified literal or an unassigned rational variable always starts from the last position. Additionally, we use the following heuristics:

\begin{itemize}
    \item \emph{Variable order}. \yaga uses a generalization of the VSIDS heuristic implementation from MiniSat~\cite{minisat}. Variable score is increased for each variable involved in conflict derivation. Variables of all types (i.e., Boolean and rational variables) are ranked using this heuristic.
    \item \emph{Restart scheme}. We use a simplified restart scheme from the Glucose solver~\cite{glucose}. The solver maintains an exponential average of glucose level (LBD) of all learned clauses~\cite{biere_2016} and an exponential LBD average of recently learned clauses. \yaga restarts when the recent LBD average exceeds the global average by some threshold.
    \item \emph{Clause deletion}. \yaga deletes subsumed learned clauses on restart~\cite{biere_2005}.
    \item \emph{Clause minimization}. Learned clauses are minimized using self-subsuming resolution introduced in MiniSat~\cite{minisat}.
    \item \emph{Value caching}. Similarly to phase-saving heuristics used in SAT solvers~\cite{darwiche_2007}, \yaga caches values of decided rational variables~\cite{jovanovic_design_2013}. It preferably uses cached values for rational variables. If a cached value is not available, the solver tries to find a small integer or a fraction with a small denominator which is a power of two.
    \item \emph{Bound caching}. We keep a stack of variable bounds for each rational variable. When the solver backtracks, it lazily removes obsolete bounds from the stack. Bounds computed at a decision level lower than the backtrack level do not have to be recomputed.
\end{itemize}
    
    
    
    
    




\section{External Code}

The solver uses a custom representation of unbounded rational values from \textsc{OpenSMT}~\cite{hyvarinen_opensmt2_2016}, which
is itself based on a library written by David Monniaux and uses GMP~\cite{GMP}.


\section{Availability}
The source code repository and more information on the solver is
available at

\begin{itemize}
    \item \url{https://github.com/d3sformal/yaga}
\end{itemize}



\bibliography{bibliography}
\bibliographystyle{plain}

\end{document}
